\section{Diskussion}
\label{sec:Diskussion}

Die frequenzabhängige Messung der Leitungskonstanten mit dem RLC-Messgerät liefert nicht das erwartete Ergebnis. Lediglich der ohmsche Belag des zweiten Kabels zeigt den erwarteten Anstieg. Der ohmsche Belag des ersten Kabels zeigt kein eindeutiges Ergebnis. Für $L$ und $C$ wird keine Frequenzabhängigkeit erwartet, diese wird jedoch gemessen.

Die Dämpfungsmessung zeigt eine sehr hohe Varianz. Teilweise wurden physikalisch nicht sinnvolle negative Dämpfungen gemessen. Der Mittelwert von $\SI{4.5}{dB}$ kann jedoch als realistisch bewertet werden.

Die durch Fit an den Signalverlauf bestimmten Leitungsparameter im dritten Aufgabenteil liegen in einer realistischen Größenordnung. Die berechneten Kabellängen

\begin{align}
  l_\mathrm{lang} &= \SI{85.94}{m} & l_{\SI{50}{\ohm}} &= \SI{9.993}{m} & l_{\SI{75}{\ohm}} &= \SI{9.993}{m} \\
\intertext{stimmen gut mit den tatsächlichen Kabellängen}
  l_\mathrm{lang, tatsächlich} &= \SI{85}{m} & l_{\SI{50}{\ohm}\mathrm{, tatsächlich}} &= \SI{10}{m} & l_{\SI{75}{\ohm}\mathrm{, tatsächlich}} &= \SI{10}{m}
\end{align}
überein.

Die bestimmten Parameter der unbekannten Bauteile liegen in einem realistischen Bereich, da sie ein Vielfaches der Leitungsparameter ohne Abschlussbauteil betragen, können allerdings nicht mit Referenzwerten verglichen werden. Die exponentiellen Fits dieser Messung können den Spannungsverlauf sehr gut modellieren.

Bei der Messung zur Mehrfachreflexion kann der erwartete Verlauf beobachtet werden, die mittleren beiden Stufen sind jedoch nicht abgegrenzt beobachtbar. Dies liegt daran, dass die Leitung einen Kapazitäts- und Induktivitätsbelag besitzt, was auch den Grund für den nicht sprunghaften Anstieg bei den anderen beiden Stufen darstellt. Die ermittelten Reflexionsfaktoren liegen teils über 1, was physikalisch nicht sinnvoll ist. Dies ist dem bereits beschriebenen Effekt zuzurechnen, der eine genaue Abmessung der Signalhöhen nicht zulässt. Insgesamt scheint es allerdings sinnvoll, dass die Reflexionsfaktoren von Anfang und Ende der Leitung nahe 1 liegen und der Faktor der Störstelle deutlich kleiner, nahe 0, ist.
