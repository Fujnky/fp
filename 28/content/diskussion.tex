\section{Diskussion}
\label{sec:Diskussion}

Der ermittelte $g$-Faktor von
\begin{align}
    g = \num{-2.12+-0.15}
\end{align}
liegt trotz der Unzulänglichkeiten unserer Durchführung nah am Literaturwert
\begin{align}
  g_\mathrm{lit} = \num[separate-uncertainty=false]{-2.00231930436182(52)}
\end{align}
aus \cite{codata}. Die händische Aufnahme der Messdaten ist insoweit problematisch, als die Brückenspannung mit der Zeit nach oben driftet. Dies fällt bei einer (schnellen) Messung mit dem XY-Schreiber nicht auf, führt aber zu Resonanzkurven, die nicht dem erwarteten Verlauf entsprechen. Möglicherweise spielen auch die Überlagerungseffekte der Resonanzkurven eine Rolle, die durch die anfängliche Verstimmung verhindert werden sollen.

Die ermittelte totale magnetische Flussdichte von
\begin{align}
  B_\bigoplus &= \SI{40+-37}{\micro\tesla}
\end{align}
liegt ebenfalls nah am Literaturwert
\begin{align}
  B_\bigoplus^\mathrm{lit} &= \SI{49.2}{\micro\tesla}
\end{align}
von \cite{magneticfield}, jedoch mit großer Standardabweichung. Auch dies ist wahrscheinlich durch die verzerrte Kurvenform begründet. Außerdem konnte mit der Bussole nicht die korrekte Ausrichtung eingestellt werden, es wurde die vorgegebene Ausrichtung beibehalten.
