\section{Ziel}
\label{sec:ziel}
Das Ziel dieses Experiments ist es, das magnetische Moment $\mu_s$ resultierend aus dem Eigendrehimpuls, dem Spin, eines freien Elektrons mit Hilfe der Hochfrequenz-Spektroskopie zu bestimmen. Desweiteren soll die totale Flussdichte des Erdmagnetfelds ermittelt werden.

\section{Theorie}
\label{sec:theorie}
\subsection{Quantenmechanischer Zusammenhang von Bahndrehimpuls und magnetischem Moment}
In der Einelektronnäherung und in Polarkoordinaten lässt sich für ein Atom folgende Wellenfunktion aufstellen
\begin{equation}
\Psi_\mathrm{n,l,m}(r,\theta,\phi) = R_{n,l}(r)\,\Theta_{l,m}(\theta)\,\Phi(\phi) = R_{n,l}(r)\,\Theta_{l,m}(\theta)\,\frac{e^{im\phi}}{\sqrt{2\pi}}.
\end{equation}
Die Wellenfunktion setzt sich aus der radialabhängigen Funktion $R$, der breitenkreisabhängigen Funktion $\Theta$ und der azimutalwinkelabhängigen Funktion $\Phi$ zusammen. Dabei bezeichnet $n$ die Hauptquantenzahl, $l$ die Bahndrehimpulsquantenzahl und $m$ die Orientierungsquantenzahl. Die drei Teile der Wellenfunktion sind normiert. Aus der Gleichung für die Stromdichte
\begin{equation}
  \vec{S}=\frac{\hbar}{2m_0i}(\Psi^* \nabla\Psi-\Psi\nabla\Psi^*)
\end{equation}
mit $\hbar=\frac{h}{2\pi}$ und $h$ als Planck'schem Wirkungsquantum und $m_0$ als Ruhemasse des Elektrons folgt, dass die radial- und breitenkreisabhängigen Anteile der Teilchenstromdichte gleich 0 sind, da die zugehörigen Wellenfunktionen reell sind. Somit ergibt sich die azimutale Teilchennstromdichte
\begin{equation}
  S_\phi = \frac{\hbar}{m_0}\frac{R^2\,\Theta^2}{2\pi}\frac{m}{r\sin\theta}.
\end{equation}
Nun soll aus dieser Stromdichte das entstehende magnetische Moment $\mu_z$ berechnet werden. Es ergibt durch Multiplikation von Kreisstrom und umschlossener Fläche.
\begin{equation}
  \mathrm{d}\mu_z=F(\theta)\mathrm{d}I_\phi
\end{equation}
Eine Integration über die Querschnittsfläche der Elektronenhülle unter Berücksichtigung der Normierungsbedingungen für die Anteile der Wellenfunktion führt zu
\begin{equation}
  \mu_z=-\frac{1}{2}{e_0}{m_0}m\hbar =:\mu_B\,m
\end{equation}
$\mu_B$ wird als Bohrsches Magneton bezeichnet.

\subsection{Energieaufspaltung beim Anlegen eines äußeren Magnetfelds}
Der Drehimpuls $\vec{l}$ kann relativ zu einer festgelegten Achse verschiedene Orientierungen einnehmen. Bezeichnet man diese Achse mit $\vec{z}$, so gilt für die $z$-Komponente des Drehimpulses
\begin{equation}
  l_z=m_l\hbar,
\end{equation}
wobei die Orientierungsquantenzahl $m_l$ alle ganzzahligen Werte zwischen $-l$ und $+l$ annehmen kann (siehe Abbildung  \ref{fig:richtungl}).
\begin{figure}
	\centering
  \includegraphics[width=0.3\textwidth] {bilder/richtungl.pdf}
	\caption{Richtungsquantelung des Bahndrehimpulses für $l=1$ \cite{anleitung28}.}
	\label{fig:richtungl}
\end{figure}
Daraus folgt, dass der Drehimpuls relativ zur Feldrichtung $2l+1$ mögliche Orientierungen besitzt.
Die Aufspaltung der Energienniveaus beim Anlegen eines äußeren Magnetfeldes $\vec{B}$ kommt dadurch zustande, dass einem magnetisches Moment $\vec{M}$ entsprechend $E=\vec{M}\cdot\vec{B}$ Energie zugeführt wird und das magnetische Moment aufgrund der Richtungsquantelung $2l+1$ unterschiedliche Werte annehmen kann. Die Aufspaltung ist äquidistant, da es sich hier um den normalen Zeeman-Effekt handelt. Hier erfolgt die Aufspaltung nach der Orientierungsquantenzahl $m_l$, da freie Elektronen betrachtet werden. Sobald die Spin-Bahn-Kopplung berücksichtigt werden muss, geschieht die Aufspaltung nach der Orientierungsquantenzahl $m_j$ ($J=L+S$). Dies wird als anomaler Zeeman-Effekt bezeichnet. In Abbildung \ref{fig:aufspaltung} ist die Energieaufspaltung durch ein äußeres Magnetfeld für ein Hüllenelektron mit $l=2$ dargestellt.
\begin{figure}
	\centering
  \includegraphics[width=0.6\textwidth] {bilder/aufspaltung.pdf}
	\caption{Aufspaltung von $E_0$ eines Hüllenelektrons mit $l=2$ durch ein äußeres Magnetfeld mit Flussdichte $B$ \cite{anleitung28}.}
	\label{fig:aufspaltung}
\end{figure}

Nicht nur der Bahndrehimpuls $l$ hat verschiedene Orientierungsmöglichkeiten. Der Eigendrehimpuls $s$ eines Elektrons hat zwei Einstellmöglichkeiten. Die zugehörige Orientierungsquantenzahl $m_s$ kann entsprechend die Werte $\pm\frac{1}{2}$ annehmen. Für die $z$-Komponente des Spins gilt
\begin{equation}
  S_z = m_s\hbar = \pm \frac{\hbar}{2}.
\end{equation}
Eine grapische Darstellung der Einstellmöglichkeiten des Spins ist in Abbildung \ref{fig:spin} zu sehen.
\begin{figure}
	\centering
  \includegraphics[width=0.3\textwidth] {bilder/spin.pdf}
	\caption{Einstellmöglichkeiten des Spins relativ zum Magnetfeld in $z$-Richtung \cite{anleitung28}.}
	\label{fig:spin}
\end{figure}

Das magnetische Moment des Elektrons, welches aus dem Spin resultiert lässt sich mit Hilfe des Landé-Faktors $g$ in Einheiten des Bohrschen Magnetons angeben. Es gilt:
\begin{equation}
  \mu_{S_z} = -g\,m_s\,\mu_B = -\frac{g}{2}\mu_B.
\end{equation}
Der Landé-Faktor beschreibt das Verhältnis vom klassisch erwarteten magnetischen Moment $\mu_B$ und dem tatsächlich gemessenen, welches aufgrund des in der klassichen Physik nicht erklärtem Spin, von der Erwartung abweicht. $g$ ist somit dimeonsionslos und liegt in der Größenordnung 1.

\subsection{Elektronenspin-Resonanz-Methode}
Ein Stoff mit freien Elektronen befindet sich in einem Magnetfeld, wodurch das Energienievau in zwei Niveaus aufgespalten wird. Die Energiedifferenz zwischen den beiden Unterniveaus berechnet sich aus
\begin{equation}
  \Delta E = g\,\mu_B\,B.
\end{equation}
Damit die Elektronen in den höheren Energiezustand übergehen, wird das System mit einer Frequenz angeregt, die dem Energieunterschied der beiden Niveuas entspicht. Dabei ändert die Orientierungsquantenzahl $m_s$ ihr Vorzeichen. Für diesen Prozess gilt die Gleichung
\begin{equation}
\label{eqn:g}
  h\,f=g\,\mu_B\,B.
\end{equation}
Dieser Effekt wird Elektronenspin-Resonanz genannt und hat zur Folge, dass die zur Anregung verwendeten "Lichtquanten" bei Resonanz absorbiert werden. Die angeregten Elektronen verlieren ihre Energie durch Wechselwirkungsprozesse mit ihrer Umgebung.
Das Magnetfeld einer Helmholtzspule lässt sich aus
\begin{equation}
\label{eqn:spule}
  B(I)=\frac{8}{\sqrt{125}}\mu_0\frac{n}{r}I
\end{equation}
berechnen. $\mu_0$ ist dabei die magnetische Feldkonstante, $n=156$ die Anzahl der Windungen, $r=\SI{0,1}{\meter}$ und $I$ der Feldstrom.
