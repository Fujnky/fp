\section{Diskussion}
\label{sec:Diskussion}

Es konnten Kontrastwerte nahe
\begin{align}
  K = \num{0.9}
\end{align}
erreicht werden, was angesichts eines maximalen Kontrasts von $K = \num{1}$ zufriedenstellend ist. Der Verlauf des Kontrasts in Abhängigkeit vom Polarisationswinkel entspricht wie erwartet annähernd einer $|\sin|$-Funktion. Die Abweichungen zwischen den Messwerten der ersten und zweiten \enquote{Halbwelle} kann nicht durch zeitliche Fluktuationen erklärt werden, da diese lediglich im Bereich von $\increment K \approx 0.001$ liegen.

Die Ausgleichsrechnung bezüglich der Glasmessung passt gut zu den Messdaten. Der bestimmte Brechungsindex
\begin{align}
  n_\mathrm{Glas} &= \num{1.544+-0.002}
\end{align}
liegt nahe am in der Anleitung angegebenen Referenzwert
\begin{align}
  n_\mathrm{Glas}^\mathrm{ref} &= \num{1.5} \text{\cite{anleitung64}}.
\end{align}
Somit kann die Messung als hinlänglich angenommen werden.

Der Luft-Brechungsindex
\begin{align}
  n_\mathrm{Luft} = \num{1.0001324+-0.0000001}
\end{align}
liegt ebenfalls nahe an der Referenz
\begin{align}
  n_\mathrm{Luft}^\mathrm{ref} &= \num{1.0003} \text{\cite{anleitung64}}.
\end{align}
Da der Brechungsindex der Luft stark von Umgebungsbedingungen wie Temperatur und Luftfeuchtigkeit abhängt, die nicht berücksichtigt werden, kann kein besseres Ergebnis erwartet werden.
