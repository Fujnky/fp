\section {Aufbau und Durchführung}
\label{sec:durchführung}

\subsection{Aufbau}
\fig{bilder/sagnac.pdf}{Aufbau eines Sagnac-Interferometers \cite{anleitung64}.}{sagnac}[width=0.7\textwidth]
In \autoref{fig:sagnac} dargstellt. Dieser besteht aus einem HeNe Laser der Wellenlänge $\lambda = \SI{632.990}{\milli\meter}$, zwei Spiegeln, die das Licht auf den polarsierenden Strahlteiler lenken und drei weiteren Spiegeln, die gemeinsam mit dem Strahlteiler ein Rechteck bilden. Die verwendete Gaszelle besitzt eine Länge von $L=(100,0 \pm 0,1)\si{\meter}$.\\
Zunächst soll die Funktionsweise eines polarisierenden Strahlteilers erläutert werden. Dieser besteht aus zwei Prismen mit den Winkeln \SI{45}{\degree}, \SI{45}{\degree} und \SI{90}{\degree}, welche an ihrer Hypotenuse miteinander verbunden sind. Der so entstehende Würfel ist in \autoref{fig:strahlteiler} zu sehen. Das auf den Strahlteiler treffende Licht wird entsprechend seiner Polarisationsrichtung aufgeteilt. Der um \SI{90}{\degree} abgelenkte Anteil ist in vertikaler Richtung polarisiert, während der Anteil mit unveränderter Richtung in horizontaler Richtung polarisiert ist. Auch dies ist in \autoref{fig:strahlteiler} zu sehen. \\
\fig{bilder/strahlteiler.pdf}{Polarsierender Strahlteiler mit eingezeichneten Achsen \cite{anleitung64}.}{strahlteiler}[width=0.6\textwidth]
Das aus dem Strahlteiler austretende Licht trifft jeweils auf einen Spiegel, welcher dieses auf einen weiteren Spiegel ablenkt, sodass es erneut auf den polarisierenden Strahlteiler trifft. Somit ergeben sich im Interferometer zwei Teilstrahlen mit unterschiedlicher Polarisation und entgegengesetzter Laufrichtung. Beim erneuten Auftreffen auf den polarsierenden Strahlteiler werden beide Strahlen so abgelenkt, dass sie auf der bisher nicht durchstrahlten Seite wieder austreten. Wird ein Polarisationsfilter hinter dieser Seite aufgestellt, gelangen die beiden Strahlen in die gleiche Polarisationsebene, sodass Interferenzeffekte möglich sind. Diese werden mittels Photodiode gemessen. Alternativ zum Polarsationsfilter kann ein zweiter polarisierender Strahlteiler verwendet werden. Der Strahl wird erneut in zwei Teilstrahlen aufgeteilt und von zwei Photodioden gemessen. Die beiden gemessenen Intensitäten werden voneinander subtrahiert. So heben sich auf beide Teilstrahlen wirkende Störungen gegenseitig auf. Dies sorgt für erhöhte Stabilität, weshalb diese Methode in diesem Versuch verwendet wird.

\subsection{Justage}
\fig{bilder/aufbau.pdf}{Sagnac-Interferometer mit eingezeichneten möglichen Positionen der Justierhilfen, sowie benannte Spiegel \cite{anleitung64}.}{aufbau}[width=0.7\textwidth]
In \autoref{fig:aufbau} ist der Aufbau mit beschrifteten Spiegeln und Positionen der Justierhilfen dargestellt, um zu verstehen, wie die Justierhilfen platziert werden müssen. Es ist wichtig, dass alle Spiegel den Strahl um jeweils \SI{90}{\degree} ablenken. Der Ablenkungswinkel kann mit einer Schraube auf der Rückseite der Spiegel verändert werden.\\
Um zu überprüfen, ob der Teilstrahl ohne Richtungsänderung gerade auf den Spiegel $M_A$ trifft, werden die Justierhilfen an den Stellen 2 und 3 positioniert und der abgelenkte Teilstrahl wird verdeckt. Anschließend wird der andere Strahlteil verdeckt und die Justierhelfen bei 8 und 9 platziert. Nun soll das Interferometer so justiert werden, dass der Spiegel $M_B$ den Strahl um \SI{90}{\degree} reflektiert. Dazu werden die Justierhilfen bei 5 und 6 angebracht. Der Strahl soll jeweils durch die mittlere Öffnung verlaufen. Nach Entfernen der Justierhilfen ist auf Spiegel $M_B$ ein Lichtpunkt zu erkennen. Außerdem soll überprüft werden, ob das Licht zwischen den Spiegeln in beide Richtungen läuft. Dazu wird ein Blatt Papier zwischen den Spiegeln $M_A$ und $M_B$ beziehungsweise $M_B$ und $M_C$ platziert. Bei korrekter Justage ist nur ein Lichtpunkt zu sehen, weil die entgegen gesetzt laufenden Teilstrahlen übereinander liegen. Nun wird ein Teilstrahl verschoben, sodass eine Probe im Strahlgang positioniert werden kann, ohne den anderen Teilstrahl zu beeinflussen. Abschließend wird dafür gesorgt, dass die aus dem polarisierenden Strahlteiler austretenden Teilstrahlen wieder übereinander liegen, damit Interferenz möglich ist. Dies ist mit den Schrauben an $M_A$ und $M_B$ möglich.

\subsection{Durchführung}
Für die Ermittlung des Brechungsindex der Glasplättchen wird die Anzahl der Interferenzmaxima in Abhängigkeit des Winkels, um den diese rotiert werden gemessen. Es wird für kleine Winkel von \SI{0}{\degree} bis \SI{10}{\degree} in \SI{2}{\degree}-Schritten jeweils drei mal gemessen, um einen Mittelwert bilden und somit den Fehler berücksichtigen zu können. Die Anzahl Anzahl der Interferenzmaxima wird dabei nach jeder Messung erneut auf null gesetzt.\\
Um den Brechungsindex der Luft zu evaluieren, wird die Gaszelle zunächst evakuiert. Anschließend wird langsam wieder Luft eingelassen und es werden im Bereich von \SI{100}{\milli\bar} bis \SI{1000}{\milli\bar} 10 Messwerte mit gleicher Druckdifferenz aufgneommen. Hier wird die Anzahl der Interferenzmaxima über den gesamten Durckbereich gemessen.
