\section {Aufbau und Durchführung}
\label{sec:durchführung}

Als Strahlungsquelle wird \isotope [241]{Am} verwendet. Zur Kollimation dienen Schlitzblenden der Größe \SI{2}{\milli\meter} $\cdot$ \SI{10}{\milli\meter}. Anschließend wird die $\alpha$-Strahlung an einer dünnen Folie gestreut und von einem Surface-Barrier Detektor in Abhängigkeit vom Streuwinkel $\theta$ aufgenommen. Die vom Detektor gemessenen Impulse werden mittels Amplifier nachverstärkt. Um Wechselwirkung der $\alpha$-Teilchen mit Luft zu verhindern, wird der Versuch im Vakuum durchgeführt. Zum Evakuieren wird eine Drehschieberpumpe verwendet. Für die Messreihe zur Ermittlung der Foliendicke wird ein Oszilloskop benutzt und zur Messung der Zählrate wird ein Zähler an den Detektor angeschlossen. Die Maße des Versuchsaufbaus können Abbildung \ref{fig:aufbau} entnommen werden.

\begin{figure}
	\centering
  \includegraphics[width=0.9\textwidth] {content/aufbau.pdf}
	\caption{Abmessungen des Aufbaus. \cite{anleitung16}}
	\label{fig:aufbau}
\end{figure}

Dicken der Folien:
\begin{itemize}
  \item Gold: $x_1=\SI{2}{\micro\meter}$, $x_2= \SI{4}{\micro\meter}$
  \item Aluminium: $x=\SI{2}{\micro\meter}$
  \item Bismut: $x=\SI{1}{\micro\meter}$
\end{itemize}


\subsection{Durchführung}
Zunächst wird die Apparatur evakuuiert. Das Oszilloskop wird angeschlossen, um die Pulse einmal mit und einmal ohne Folie beobachten zu können.

Um die Aktivität zu bestimmen, wird die Zählrate (Ereignisse pro Zeit) ohne Folie gemessen.

Zur Bestimmung der Foliendicke wird eine Energieverlustmessung durchgeführt. Dabei ist der Streuwinkel $\theta=0°$. Die Pulshöhe wird in Abhängigkeit des Drucks in der Streukammer gemessen. Die Messung erfolgt einmal mit und einmal ohne die Folie.

Um den differentiellen Wirkungsquerschnitt bestimmen zu können, wird die Zählrate in Abhängigkeit vom Streuwinkel gemessen. Dabei ist zu beachten, dass das Zeitintervall, in dem gemessen wird, für größere Winkel länger sein muss, um ausreichend viele Ereignisse zu detektieren.


Für die Messung zum Einfluss der Mehrfachstreuung wird die Zählrate für beide Goldfolien unter einem festen Winkel ($\theta=10°$) gemessen.

Zur Untersuchung der $Z$-Abhängigkeit soll die Intensität der $\alpha$-Teilchen für die verschiedenen Folien unter einem großen Streuwinkel gemessen werden. Diese Messreihe entfällt hier, da die Folien aus Aluminium und Bismut nicht richtig in der Apparatur angebracht werden konnten.
