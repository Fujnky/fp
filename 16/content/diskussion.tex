\section{Diskussion}
\label{sec:Diskussion}

\subsection{Aktivität}
Die deutliche Abweichung der aus dem Zerfallsgesetz bestimmten Aktivität gegenüber der gemessenen ist vermutlich auf Unzulänglichkeiten des Versuchsaufbaus zurückzuführen: Die Akzeptanz des Detektors wird nicht berücksichtigt und es ist nicht bekannt, wie genau die reale Geometrie des Aufbaus mit der Versuchsanleitung übereinstimmt. Daraus folgt ein möglicher Fehler bei der Bestimmung des Raumwinkels. Des Weiteren kann eine exakt rechtwinklige Ausrichtung des Strahlers nicht gewährleistet werden.

\subsection{Foliendicke}
Die bestimmte Foliendicke von $\SI{3.84}{\micro\meter}$ unterscheidet sich signifikant von der angegebenen Dicke von $\SI{2}{\micro\meter}$. Dies kann möglicherweise durch eine nicht ganz rechtwinklig zum Strahlengang angebrachte Folie erklärt werden, andererseits ist diese selbst nicht (mehr) ganz plan. Das gleichzeitige Ablesen von Druck und gemittelter Spannung ist nur eingeschränkt möglich, da beide sich nicht stabil verhalten. Die Bethe-Formel enthält einen Geschwindigkeitsterm $v$, dessen Änderung im Laufe des Materiedurchtritts von uns nicht berücksichtigt wird.

\subsection{Rutherford-Streuung}
In \autoref{fig:plot3} ist erkennbar, das unser gemessener differentieller Wirkungsquerschnitt einen der Theoriekurve annähernd ähnlichen Verlauf vorweist. Im Gegensatz zu dieser ist er jedoch nicht \enquote{steil genug}. Dass die Messwerte für kleine Winkel nicht gegen unendlich gehen ergibt sich aus der endlichen Aktivität der Quelle. Zu wenig Statistik durch kurze Integrationszeiten scheint nicht der Grund für die Abweichung zu sein, da keine großen Schwankungen im Messdatenverlauf zu erkennen sind. Weiterhin findet keine Hintergrund-Messung (des Strahlprofils) statt, durch die die gestreuten von den ungestreuten Signalen unterschieden werden könnten.

\subsection{Mehrfachstreuung}
Anhand der geringeren Zählrate für die dickere Folie ($\SI{1.159}{\per\second}$ gegenüber $\SI{1.368}{\per\second}$) lässt sich ableiten, dass dort Mehrfachstreuung in nicht vernachlässigbarer Häufigkeit stattfindet. Durch einen längeren Weg innerhalb der Folie finden mehr Wechselwirkungen statt, durch die mehr Teilchen absorbiert werden.
