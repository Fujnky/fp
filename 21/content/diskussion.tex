\section{Diskussion}
\label{sec:Diskussion}

\subsection{g-Faktoren}

Die gemessenen Resonanzflussdichten lassen sich durch die lineare Regression gut modellieren. Die angegebenen Fehler beziehen sich lediglich als statistische Fehler auf die Ausgleichsgerade, die tatsächliche Messungenauigkeit liegt aufgrund von unberücksichtigten systematischen Fehlern höher. Als Vergleich dienen die mithilfe von \eqref{eqn:näherung} berechneten Theoriewerte
\begin{align}
  g_{F,\mathrm{th}}^{\ce{^{85}Rb}} &\approx \frac13 \\
  \intertext{und}
  g_{F,\mathrm{th}}^{\ce{^{87}Rb}} &\approx \frac12,
\end{align}
welche in guter Übereinstimmung mit den gemessenen Werten stehen. \\
Die berechneten Werte der Kernspins stimmen nicht mit den Literaturwerten aus \cite{rubidium}
\begin{align}
  I_{\ce{^{85}Rb}} &= \frac52 \\
  \intertext{und}
  I_{\ce{^{87}Rb}} &= \frac32
\end{align}
überein, was nicht erklärt werden kann.

\subsection{Erdmagnetfeld}

Als Vergleich für die Komponenten des Erdmagnetfelds dient \cite{magneticfield} mit
\begin{align}
  B^\oplus_\mathrm{vert, Modell} &= \SI{45.2}{\micro\tesla} \\
  B^\oplus_\mathrm{hor, Modell} &= \SI{19.3}{\micro\tesla}.
\end{align}

Diese Werte stimmen ebenfalls in zufriedenstellender Genauigkeit mit unseren Messwerten überein. Ein möglicher Fehler bezüglich der Horizontalkomponente ist eine nicht exakte Nord-Süd-Ausrichtung des Versuchsaufbaus. Die Vertikalkomponente hingegen ist durch die schwierige Justage des Spulenstroms fehlerbehaftet. Aufgrund des Versuchsstandortes innerhalb eines Gebäudes (aus u.a. Stahl) sind allerdings ohnehin Perturbationen des magnetischen Feldes zu erwarten.

\subsection{Quadratischer Zeeman-Effekt}
Wie schon in \autoref{sec:QuadZee} erwähnt, hat der quadratische Zeeman-Effekt bei den verwendeten Magnetfeldstärken keinen signifikanten Einfluss. Bei der vorhandenen Messgenauigkeit macht sich dieser Effekt erst bei höheren Feldstärken bemerkbar.
