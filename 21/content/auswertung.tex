\section{Auswertung}
\label{sec:Auswertung}
\subsection{Vertikalfeld}
Zur Kompensation des vertikalen Erdmagnetfelds ist eine Stromstärke von
\begin{align}
  I_\mathrm{vert} = \SI{0.32}{\ampere}
\end{align}
nötig, was über
\begin{equation}
  B(I)=\frac{8}{\sqrt{125}}\mu_0\frac{n}{r}I
\end{equation}
mit den zuvor genannten Spuleneigenschaften einer magnetischen Flussdichte von
\begin{align}
  B^\oplus_\mathrm{vert} = \SI{35.25}{\micro\tesla}
\end{align}
entspricht.

\subsection{Gyromagnetische Faktoren}
Für die verschiedenen Frequenzen der HF-Spule wurden die in \autoref{tab:feld} aufgelisteten Stromstärken der Resonanzstellen gemessen, sowie analog die magnetische Flussdichte bestimmt. In \autoref{fig:plot1} werden die Werte einschließlich linearer Regression grafisch dargestellt. Der gemittelte Ordinatenabschnitt beider Geraden entspricht der Horizontalkomponente des Erdmagnetfelds und wurde zu
\begin{align}
  B^\oplus_\mathrm{hor} &= \SI{21.38+-0.12}{\micro\tesla}
\end{align}
bestimmt. Aus \eqref{eqn:gf} folgt, dass aus der Steigung $m$ der Geraden über
\begin{align}
  g_F = \frac{h}{m \mu_b}
\end{align}
der gyromagnetische Faktor des jeweiligen Isotops folgt. So wurden die Faktoren
\begin{align}
  g_F^{\ce{^{85}Rb}} &= 0{,}330\pm0{,}0000000011 \\
  \intertext{und}
  g_F^{\ce{^{87}Rb}} &= 0{,}495\pm0{,}0000000024
\end{align}
ermittelt. Über \eqref{eqn:näherung} folgt der Kernspin
\begin{align}
  I_{\ce{^{85}Rb}} &= 0{,}509\pm0{,}000000005 \\
  \intertext{und}
  I_{\ce{^{87}Rb}} &= 1{,}014\pm0{,}0000000005
\end{align}

\input{build/feld.tex}
\fig{build/plot1.pdf}{Gemessene magnetische Flussdichten, aufgetragen über die zugehörige Frequenz.}{plot1}


\subsection{Isotopenverhältnis}
Anhand der Höhe der Resonanzkurven kann das Isotopenverhältnis der verwendeten Rubidiumprobe abgeschätzt werden. Dazu wird bei einer Frequenz von $\nu = \SI{100}{kHz}$ das Magnetfeld durchfahren und die Resonanzkurve mithilfe eines Oszilloskops aufgenommen. Da das im Versuch verwendete Oszilloskop im XY-Modus keinen Datenexport erlaubt, werden beide Kanäle im YT-Modus mithilfe eines Single-Shot-Triggers festgehalten und dann einzeln abgespeichert. Das Ergebnis findet sich in \autoref{fig:plot2}. Die Resonanzstellen werden mit einer Cauchy-Verteilung gefittet, sodass die Extremstelle jeweils präzise ermittelt werden kan. Letztendlich ergibt sich ein Isotopenverhältnis von
\begin{align}
  \frac{N(\ce{^{87}Rb})}{N(\ce{^{85}Rb})} \approx \frac{\num{2.42}}{\num{3.85}} \approx \num{0.63}
\end{align}
\fig{build/plot2.pdf}{Verlauf der Intensität und Stromstärke in Abhängigkeit von der Zeit bei einer Messung mit $\nu=\SI{100}{\kilo\hertz}$. Fits der Resonanzstellen mit einer modifizierten Cauchy-Verteilung.}{plot2}

\subsection{Quadratischer Zeeman-Effekt}
\label{sec:QuadZee}
In \autoref{fig:quadratisch} sind die Messwerte erneut dargestellt, wobei die lineare Regression jeweils abgezogen wurde. Damit bleiben lediglich Terme höherer Ordnung übrig. Da keine Parabel oder Ähnliches erkennbar ist, kann davon ausgegangen werden, dass der quadratische Zeeman-Effekt bei den verwendeten Flussdichten einen so geringen Effekt hat, dass er im Rahmen unserer Messgenauigkeit nicht erfasst werden kann.
\fig{build/plot3.pdf}{Residuen der linearen Regression aus \autoref{fig:plot1}, also die Differenz der Messwerte zur linearen Regression.}{quadratisch}
