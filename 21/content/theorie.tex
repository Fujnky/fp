\section{Ziel}
\label{sec:Ziel}

Im Versuch sollen die Landéschen $g_f$-Faktoren der Rubidium-Isotope $\ce{^{85}Rb}$ und $\ce{^{87}Rb}$, deren Kernspins sowie das Isotopenverhältnis in der untersuchten Probe bestimmt werden. Weiterhin soll der Einfluss des quadratischen Zeeman-Effekts auf die Messung untersucht werden.

\section{Theorie}
\label{sec:theorie}

\subsection{Drehimpulse und magnetische Momente}
In der Quantenmechanik induzieren Drehimpulse geladener Teilchen magnetische Momente. Das magnetische Moment der Elektronenhülle eines Atoms etwa ist über
\begin{align}
  \vec \mu_J = -g_J \mu_B \vec J
\end{align}
mit dem Gesamtdrehimpuls der Hülle $\vec J$ verknüpft. Dabei ist $g_J$ der gyromagnetische Faktor, welcher das Verhältnis zwischen klassich erwartetem und quantenmechanischem Zusammenhang beschreibt. Dieser ergibt sich aus der Berücksichtigung der Spin-Bahn-Kopplung, die mangels Spin in der klassischen Erwartung natürlich nicht berücksichtigt wird. Zusätzlich zu Elektronenspin $\vec S$ und Bahndrehimpuls $\vec L$ muss beim Atom noch der Kernspin $\vec I$ berücksichtigt werden, sodass für den Gesamtdrehimpuls des Atoms dann
\begin{align}
  \vec F = \vec J + \vec I
  \intertext{mit}
  \vec J = \vec L + \vec S
\end{align}
folgt. Wird nun ein externes Magnetfeld $\vec B$ angelegt, ergibt sich durch die magnetischen Dipolmoment eine potentielle Energie
\begin{align}
  E_\mathrm{pot} = - \vec \mu_F \cdot \vec B.
\end{align}
Da gemäß der Quantenmechanik der Drehimpuls gequantelt ist, gibt es nur eine abzählbare Anzahl von möglichen Energiezuständen. Dies führt letztendlich zu der in \autoref{fig:aufspaltung} dargestellten Aufspaltung der Energieniveaus, die wir uns in diesem Experiment zunutze machen.

\fig{bilder/aufspaltung.pdf}{Energieniveaufspaltung eines Atoms durch ein angelegtes externes Magnetfeld.}{aufspaltung}[width=0.6\textwidth]
